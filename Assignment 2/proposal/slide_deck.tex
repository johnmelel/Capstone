\documentclass{beamer}
\usepackage[utf8]{inputenc}
\usepackage{graphicx}

\title{Utilizing RAG LLM Technology for Centralized Access to Revolutionize Medical Paper Retrieval}
\author{Generated by Unified Research Proposal Agent}
\date{\today}

\begin{document}

\frame{\titlepage}

\begin{frame}
\frametitle{Overview}
\tableofcontents
\end{frame}

\begin{frame}
\frametitle{Abstract}
This research proposal aims to develop a Retrieval-Augmented Generation (RAG) Language Model (LLM) for the retrieval of medical papers, enabling a centralized vector store to efficiently pull papers, articles, and journals. The primary goal of this proposal is to enhance the accessibility of relevant medical literature for researchers and healthcare professionals, thereby improving the efficiency and effectiveness of academic research in healthcare.
\end{frame}

\begin{frame}
\frametitle{Background \& Literature Review}
Background:
Efficiently accessing relevant medical literature is essential for researchers to stay informed about the latest advancements in the field. Traditional methods of searching for medical literature, such as keyword searches in databases like PubMed or Google Scholar, have limitations in terms of precision, recall, and time consumption. As the volume of medical literature continues to grow exponentially, researchers face the challenge of keeping up with the vast amount of information available, necessitating the development of more advanced and efficient information retrieval methods in the medical field.
\end{frame}

\begin{frame}
\frametitle{Problem Statement \& Research Gap}
Problem Statement:
The current retrieval systems for medical literature are inefficient and time-consuming, requiring users to manually search through multiple databases and sources. This process is labor-intensive and error-prone, exacerbated by the lack of a centralized vector store for academic literature. As a result, researchers struggle to access and analyze information effectively.
\end{frame}

\begin{frame}
\frametitle{Proposed Gen AI Approach}
Proposed Gen AI Approach:
The proposed approach aims to develop a Retrieval-Augmented Generation (RAG) LLM for efficiently retrieving medical papers by integrating state-of-the-art language models with a centralized vector store. The architecture of the RAG LLM will consist of three main components: a retrieval model, a generation model, and a vector store integration module.
\end{frame}

\begin{frame}
\frametitle{Expected Impact in Healthcare}
The implementation of a Retrieval-Augmented Generation (RAG) LLM for the retrieval of medical papers is poised to have a profound impact on healthcare research. This technology will facilitate the creation of a centralized vector store capable of efficiently pulling papers, articles, and journals en masse, thereby transforming the approach researchers take to accessing and analyzing medical literature.
\end{frame}

\begin{frame}
\frametitle{Limitations / Ethical Considerations}
Limitations and Ethical Considerations:
Data Privacy: A key ethical consideration in this research proposal is the issue of data privacy. To create a Retrieval-Augmented Generation (RAG) LLM for medical paper retrieval, a centralized vector store will need to collect and store a significant amount of data, including sensitive medical information.
\end{frame}

\begin{frame}
\frametitle{System Workflow and User Interaction}
\begin{figure}
% \includegraphics[width=0.8\textwidth]{system_workflow.png}
\caption{System Workflow Diagram}
\end{figure}
\end{frame}

\begin{frame}
\frametitle{Conclusion / Future Work}
In conclusion, the proposed Gen AI approach aims to develop an advanced system for retrieving medical papers by leveraging language models and centralized vector stores. This system has the potential to transform how researchers access and retrieve medical information, leading to faster and more efficient knowledge discovery in the field of medicine.
\end{frame}

\end{document}